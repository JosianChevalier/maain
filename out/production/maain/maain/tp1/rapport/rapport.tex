%%%%%%%%%%%%%%%%%%%%%%%%%%%%%%%%%%%%%%%%%
% University/School Laboratory Report
% LaTeX Template
% Version 3.1 (25/3/14)
%
% This template has been downloaded from:
% http://www.LaTeXTemplates.com
%
% Original author:
% Linux and Unix Users Group at Virginia Tech Wiki 
% (https://vtluug.org/wiki/Example_LaTeX_chem_lab_report)
%
% License:
% CC BY-NC-SA 3.0 (http://creativecommons.org/licenses/by-nc-sa/3.0/)
%
%%%%%%%%%%%%%%%%%%%%%%%%%%%%%%%%%%%%%%%%%

%----------------------------------------------------------------------------------------
%	PACKAGES AND DOCUMENT CONFIGURATIONS
%----------------------------------------------------------------------------------------

%&pdflatex
\documentclass{article}

\usepackage[utf8]{inputenc}
\usepackage[french]{babel}
\usepackage[version=3]{mhchem} % Package for chemical equation typesetting
\usepackage{siunitx} % Provides the \SI{}{} and \si{} command for typesetting SI units
\usepackage{graphicx} % Required for the inclusion of images
\usepackage{natbib} % Required to change bibliography style to APA
\usepackage{amsmath} % Required for some math elements 

\setlength\parindent{0pt} % Removes all indentation from paragraphs

\renewcommand{\labelenumi}{\alph{enumi}.} % Make numbering in the enumerate environment by letter rather than number (e.g. section 6)

%\usepackage{times} % Uncomment to use the Times New Roman font

%----------------------------------------------------------------------------------------
%	DOCUMENT INFORMATION
%----------------------------------------------------------------------------------------

\title{Moteurs de recherche \\ TP1} % Title

\author{Josian \textsc{Chevalier}\\Thomas \textsc{Salmon}} % Author name

\date{\today} % Date for the report

\begin{document}

\maketitle % Insert the title, author and date

\section{Exercice 1}

	Le type matrice peut être initialisé de deux façons :
	\begin{itemize}
		\item En passant en paramètre de son constructeur les tableaux C, L et I. Dans ce cas elle sera directement initialisée avec les bonnes valeurs mais cela implique de calculer C, L et I au préalable
		\item En passant en paramètre de son constructeur la taille de la matrice, auquel cas toutes les cases seront initialisées à 0. On pourra modifier leurs valeurs à la de la fonction changeValue(), afin de modifier les valeurs case par case.
	\end{itemize}

	L'initialisation d'une matrice est en temps constant. La récupération ou la modification d'une valeur est en O(n), car on parcours au maximum une sous section de I, qui représente une ligne de la matrice.

	La multiplication par un vecteur se fait à l'aide de la fonction mult\_vect(). Elle parcours L (de taille n), et pour chaque ligne parcours la section de C et I qui correspond à la ligne. Ainsi on parcours Une fois L, et une seule fois chaque valeur de I (et de C). La complexité de cette fonction est donc de O(n+m).

\section{Exercice 2}

	Ce calcul s'effectue au travers de la fonction mult\_vect\_transp(). L'algorithme est le même que pour le calcul de la multiplication par vecteur, on change simplement les éléments du vecteurs résultats affectés à chaque boucle. Ainsi, on garde une complexité en O(n+m).

\section{Exercice 3}

	Pour faire un algorithme calculant la probabilité d'être sur les différents sommets en un certain nombre de pas, nous avons légèrement modifié PageRank. Ainsi sa condition d'arrêt n'est plus l'atteinte de la marge d'erreur acceptable, mais on boucle simplement autant de fois que l'on souhaite avoir d'étapes.
	
	La classe PageRank contient les méthodes permettant d'effectuer ces traitements :
	
	\begin{itemize}
		\item La méthode `rank()' permet de partir de la liste des probabilité de se trouver sur chaque sommet, et calcule pour un nombre de pas les probabilités en fonction d'une matrice.
		\item La méthode `rank\_from\_sommet()' utilise la méthode précédentes en lui fournissant une liste ou  la probabilité d'être sur chaque sommet est de 0, sauf pour le sommet `num\_sommet', pour laquelle elle est de 1.
		\item La méthode `rank\_zero()' utilise la méthode précédente en lui indiquant que l'on part du sommet 0.
	\end{itemize}

\section{Exercice 4}



	 

\end{document}